\documentclass[main.tex]{subfiles}
\begin{document}

\chapter*{Introduction Générale}
\addcontentsline{toc}{chapter}{Introduction Générale}

L'agriculture moderne et les marchés financiers sont deux domaines qui, bien que distincts, partagent un besoin commun crucial : l'accès rapide et précis à l'information pour la prise de décision. Dans le contexte marocain et international, les agriculteurs sont confrontés à des défis croissants liés à la gestion des cultures, à la détection précoce des maladies et à l'optimisation des ressources. Parallèlement, l'investissement boursier nécessite des outils d'analyse performants pour prédire les tendances du marché.

Le projet \textbf{AgroSense} naît de la volonté de combiner ces deux mondes à travers une solution technologique innovante. Il s'agit d'une application mobile multi-facettes développée en Flutter, intégrant des technologies d'Intelligence Artificielle de pointe. AgroSense ne se contente pas d'être un simple outil de gestion ; elle se positionne comme un assistant intelligent capable de diagnostiquer les plantes, de prédire les cours de la bourse, et d'interagir naturellement avec l'utilisateur via un assistant vocal et un système de génération augmentée de récupération (RAG).

Cependant, la puissance de ces fonctionnalités nécessite un niveau de sécurité élevé. C'est pourquoi un accent particulier a été mis sur la \textbf{sécurité biométrique}. L'accès à l'application est verrouillé par un système de reconnaissance faciale robuste, développé "from scratch" en utilisant des réseaux de neurones siamois (Siamese Neural Networks). Ce choix garantit que les données sensibles et les fonctionnalités critiques de l'application restent protégées et accessibles uniquement aux utilisateurs autorisés.

Ce rapport de Projet de Fin d'Année décrit le processus complet de conception et de réalisation d'AgroSense. Nous commencerons par présenter le cadre du projet et l'étude de l'existant dans le premier chapitre. Le deuxième chapitre détaillera les spécifications des besoins fonctionnels et techniques. Le troisième chapitre sera consacré à la conception du système, incluant les diagrammes UML et l'architecture détaillée, avec un focus majeur sur le module de reconnaissance faciale. Enfin, le quatrième chapitre présentera la réalisation technique, les interfaces de l'application et les résultats obtenus.

\end{document}
