\documentclass[main.tex]{subfiles}
\begin{document}

\chapter*{Conclusion Générale}
\addcontentsline{toc}{chapter}{Conclusion Générale}

Le projet \textbf{AgroSense} nous a permis de concrétiser une vision ambitieuse : réunir l'agriculture de précision et l'analyse financière dans une seule application mobile sécurisée. Au terme de ce travail, nous avons réussi à développer une solution complète qui répond aux besoins des agriculteurs et des investisseurs modernes.

L'aspect le plus critique et le plus innovant de ce projet a été la mise en œuvre de la \textbf{reconnaissance faciale via des réseaux de neurones siamois}. En développant cette brique de sécurité "from scratch", nous avons acquis une maîtrise approfondie des mécanismes d'apprentissage profond (Deep Learning) et des défis liés à l'authentification biométrique (One-Shot Learning, gestion des seuils de similarité). Ce choix architectural garantit une indépendance technologique et une protection accrue des données utilisateurs.

Au-delà de la sécurité, l'intégration de modèles prédictifs pour les maladies des plantes (CNN) et pour le marché boursier (LSTM), ainsi que l'ajout d'un assistant intelligent (RAG), font d'AgroSense un véritable assistant personnel polyvalent.

Ce Projet de Fin d'Année a été une opportunité inestimable pour monter en compétence sur le framework Flutter et l'écosystème Python/TensorFlow. Il ouvre la voie à de futures améliorations, telles que le déploiement des modèles directement sur le mobile (Edge AI) pour réduire la dépendance à la connexion internet, ou l'élargissement de la base de données de maladies agricoles.

\end{document}
